\documentclass[a4paper, 14pt, titlepage, fleqn]{extarticle}
\usepackage{style}
\usepackage{amsfonts}
\usepackage{float}
\usepackage{esint}
\usepackage{indentfirst}
\usepackage{setspace}

\begin{document}
	\fefutitlepage{ОТЧЕТ}{к лабораторной работе №1\\по дисциплине <<Математическое моделирование>>}{01.03.02 <<Прикладная математика и информатика>>}{Б9120-01.03.02миопд}{Крюков Н.В.}
	\tableofcontents
	\newpage

    \onehalfspacing
    
    \sect{Введение}

        В данной лабораторной работе я буду решать задания, используя программы компьютерной математики. Оформлять решенные задачи буду в среде компьютерной верстки <<\TeX>>, затем конвертировать в документ формата PDF.
        
    \sect{Задача о выборе транспортного средства}

        \subsect{Постановка задачи}
            Внезапно возникла потребность. Стало необходимо разгоняться на автомобиле с нуля до определённой скорости за определённое время. А также выбрать автомобиль, мощности которого хватит разогнаться за такое время. 
        
        \subsect{Выбор переменных}
            Множество автомобилей и необходимых условий можно охарактеризовать конкретными параметрами
            
            \begin{enumerate}
                \item массой $m$;
                \item скоростью $v$, до которой должен разогнаться автомобиль;
                \item временем $\tau$, за которое автомобиль должен разогнаться до нужной скорости;
                \item минимальной мощностью $P_{min}$, которой должен обладать автомобиль для вышеперечисленных условий.
            \end{enumerate}

            % \begin{enumerate}
            %     \item массой $m = 1 \ \text{тонна}$;
            %     \item скоростью $v = 100 \ \text{км/ч} $, до которой должен разогнаться автомобиль;
            %     \item временем $\tau = 2 \ \text{секунды}$, за которое автомобиль должен разогнаться до нужной скорости;
            %     \item минимальной мощностью $P_{min}$, которой должен обладать автомобиль для вышеперечисленных условий.
            % \end{enumerate}
        \subsect{Выбор законов и зависимостей}
            Для того, чтобы узнать, какая минимальная мощность требуется для автомобиля, нужно понять, как этот параметр влияет на остальные.
            Мощность автомобиля влияет на работу этого транспортного средства.
            Работа автомобиля за определённый промежуток времени равна произведению мощности автомобиля на время работы -- \(A = P \cdot \tau\). 
            
            Работа автомобиля будет уходить на разгон, на набирание скорости, то есть на увеличение кинетической энергии.
            Формула кинетической энергии -- \(A = \dfrac{m v^2}{2}\).
            Действием остальных сил, действующих на разгоняющийся автомобиль, будем пренебрагать.

        \subsect{Формулировка математической модели}
            Приравняем вышеприведённые уравнения друг к другу:
            \[P \cdot \tau = \dfrac{m v^2}{2}\]
            Следовательно, формула минимальной мощности, которой должен автомобиль для разгона с нуля до скорости v равна:
            \[P = \dfrac{m v^2}{2 \cdot \tau}\]
            
            Математическая модель поставлена

        \subsect{Решение}
            Решим задачу с конкретными параметрами
            
            \begin{enumerate}
                \item $m = 1 \ \text{тонна}$;
                \item $v = 100 \ \text{км/ч} $;
                \item $\tau = 2 \ \text{секунды}$.
            \end{enumerate}

            Переведём все значения в систему СИ:

            \[\begin{split}
                & 100 \ \text{км/ч} = 100 \cdot \dfrac{1000}{3600} \ \text{м/с} = 27 \dfrac{7}{9} \ \text{м/с} \\
                & 1 \ \text{тонна} = 1000 \ \text{кг}
            \end{split}\]

            Подставив все значения в формулу, получим:

            \[ P = \dfrac{ 1000 \cdot \left(27 \tfrac{7}{9}\right)^2}{2 \cdot 2} = 69 \ 444 \dfrac{4}{9} \ \text{Вт} \approx 70 \ \text{кВт} \approx 95 \ \text{л.с.}\]
            
    \sect{Заключение}
        При мощности 95 л.с. автомобиль массой 1 тонна разгонится до 100 км/ч за 2 секунды.

\end{document}